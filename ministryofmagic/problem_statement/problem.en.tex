\problemname{Ministry of Magic}
As you may know, many countries have a system of government where the parliment chooses the prime minister, who then forms a government.
Generally, the parliment needs to elect the prime minister by majority vote.
Since no single party generally get a majority in parliment, parties have to form a coalition that gets a majority vote.
As a result, some countries have went by without a government for years during negotiations!

In the wizarding world, one quite early came to the conclusion that lacking a government for years is not sustainable.
Dark wizards could easily use the lack of government to take over the wizarding world without anyone to fight back.
Thus, election of minister of magic was early on decided to guarantee that an election completed as soon as everyone cast their votes.
The system they use for this is called \emph{instant-runoff voting}.

In instant-runoff voting, $C$ candidates compete against each other in up to $C - 1$ rounds.
After each round, if one of the candidates get a majority vote, that candidate is elected.
Otherwise, the least popular candidate is eliminated.
If two candidates both are equally unpopular, the lowest numbered one is eliminated.
Here, a majority vote requires a \emph{strict} majority, i.e. more than half of the votes.

Votes for all the rounds are specified before the first ronud by giving a ranking of all the $C$ candidates.
Given such a ranking, a voter will vote for the highest ranked candidate in their ranking that haven't been eliminated.
For example, if a voter has the ranking $4, 1, 2, 3, 5$ and the candidates $1, 4, 5$ have been eliminated, the voter will vote for candidate $2$ in the next round.

Just as in the normal world, the wizarding world have a number of parties (more specifically $P$ parties).
Each party have a number of representatives in the parltiment that chooses the minister.
You will be given the candidate rankings for each party.
Can you determine who will win the election?

\section*{Input}
The first line contains two integers -- the number of candidates $1 \le C \le 1000$ and parties $1 \le P \le 1000$.
The next $P$ lines describes each party.
Each line starts with an integer $v$ ($1 \le v \le 10^6$), the number of votes that the party have in parliment, followed by a permutation of the numbers $1, 2, \dots C$, party's ranking of the candidates.
The ranking is given from most preferred candidate to least preferred candidate.

\section*{Output}
Output a single integer, the number of the candidate that is elected.
